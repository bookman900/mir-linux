\documentclass[12pt]{book}
\pagestyle{headings} % для более удобной навигации
\usepackage[utf8]{inputenc} % чтобы не глючило. Собирается через pdfTeX!!!
\usepackage[english,russian]{babel} % 2 языка
\usepackage[margin=2cm]{geometry} % стандартная ширина поля в книгах
\usepackage{setspace} % полуторный интервал
\usepackage{tcolorbox} % будет полезно
\usepackage{color} % будет полезно
\usepackage{droid} % вполне неплохо смотрится
\usepackage{graphicx}
\usepackage{pdfpages} % для вставки pdf при необходимости
\usepackage{hyperref}
\usepackage{indentfirst} % для отступов в основном тексте

%%% Поле для настройки отдельных %%%
%%  Элементов	ДО текста          %%
%\renewcommand{\abstractname}{О Книге}
%%%%%%%%%%%%%%%%%%%%%%%%%%%%%%%%%%%%
\begin{document}
%%% Поле для настройки отдельных %%%
%%  Элементов	по ходу             %%

%%% Оформление титульного листа %%%%
%   Титульный лист будет присое-  %%
%   диняться к файлу через pdf-   %%
%   shuffler, чтобы сработали     %%
%   все необходимые ссылки			  %%
%%%%%%%%%%%%%%%%%%%%%%%%%%%%%%%%%%%%

%%%   Страница с информацией     %%%
\pagestyle{empty}
\textbf{Мир Linux}

\phantom{}
\textbf{Свен Вермален}

\phantom{}
Copyright © 2009-2013 Sven Vermeulen

\phantom{}
Перевод с английского: Галым Керимбеков\footnotetext{\href{kerimbekov.galym@yandex.ru}{$^{*}$kerimbekov.galym@yandex.ru}}$^{*}$

\phantom{}
Пост-верстка: Stas Bookman\footnotetext{\href{nodor4mint@yandex.ru}{$^{**}$nodor4mint@yandex.ru}}$^{**}$
%%% Аннотация %%%

\phantom{}

\phantom{}
\begin{tcolorbox}[colback=blue!12!white]
\noindent Книга "Мир Linux" предлагает мягкое, но не лишенное технических аспектов (с точки зрения конечного пользователя) знакомство с операционной системой на примере дистрибутива Gentoo Linux. Книга не станет описывать историю ядра Linux или дистрибутивов, либо погружать в детали, менее интересные для пользователей.

\phantom{}
\noindent Онлайн-руководство Gentoo предлагает весьма подробный подход по ряду разделов и поэтому обязательно к чтению любым пользователем, желающим знать обо всех возможностях этой операционной системы. Несмотря на то, что издание "Мир Linux"{} и онлайн-руководство Gentoo определённо пересекаются между собой, книга ни в коем случае не призвана заменить последнее.

\phantom{}
\noindent "Мир Linux"{} попытается сосредоточиться на темах, о которых повседневные пользователи, вероятно, должны знать, чтобы продолжать работать с Gentoo Linux.

\phantom{}
\noindent Версия, которую вы читаете в настоящее время, является v1.17 и была сгенерирована 18.02.2016 г. Доступны также версии ODT и ePUB.
\end{tcolorbox}

%%%  Оглавление  %%%
\tableofcontents
%%%%%%%%%%%%%%%%%%%%

\newpage

%%%  Введение  %%%

\onehalfspacing % дальше по тексту -- полуторный интервал
%\pagestyle{headings} % С номерами страниц и заголовками

\section*{Введение}

В области настольных графических сред, Linux, предположительно, небольшой игрок (рыночные исследования оценивают долю на рынке приблизительно в 3\%). Однако наиболее вероятно, что вы знаете двух или более человек, использующих Linux, некоторые из них даже исключительно. Если принять это во внимание, то либо вы знакомы с предпочтениями ОС у более, чем ста человек, либо данную статистику стоит рассматривать с некоторым сомнением.

Тем не менее, 3\% - это все ещё много (Вы когда-нибудь думали о том, насколько много настольных систем? Я никогда не находил ответа на  этот вопрос). И если мы примем во внимание другие рынки (встраиваемые системы, серверы, сетевые устройства и прочее), доля Linux увеличится.

Но тем не менее, множество людей понятия не имеют, что такое Linux или как с ним работать. В этой книге я предлагаю техническое, краткое введение в операционную систему  с пользовательской точки зрения. Я не собираюсь погружаться в понятия, преимущества или недостатки Свободного Программного обеспечения (тем не менее, несколько абзацев не станут болезненными) и рассказывать об истории и развитии операционных систем Linux. Для ознакомления с большим числом ресурсов по этим темам, обратитесь к разделу «Дальнейшие ресурсы» в конце этой главы.

Чтобы стать полноценной книгой о Linux как об операционной системе, важно сообщить пользователю об операционных системах в целом. Linux очень модульный и открытый, и это подразумевает, что пользователю виден каждый компонент в системе. Без понимания структуры ОС пользователю было бы трудно осмыслить предназначение каждого модуля. Поэтому, я посвящаю весь раздел операционным системам.

Как только мы дойдём до задач операционной системы, я продолжу рассказ о реальных системах Linux: дистрибутивах.

В завершение, каждая глава в этой книге предложит ряд упражнений, которые можно попытаться решить. Вы не сможете найти в этой книге ответы по каждому вопросу. Предпочтительнее было бы взглянуть на упражнения, как на средство для дальнейшего продвижения и помощь в поиске (и нахождении) больше тем, связанных с Linux. В конце книги приведён список подсказок и/или ответов по вопросам.

\newpage

%%% Глава 1: Анатомия операционной системы %%%

\chapter{Анатомия операционной системы}

Операционная система - фактически стек программного обеспечения, каждый элемент, разработанный для определённой цели.

\begin{itemize}
	\item Ядро системы: управляет обменом данных между устройствами и программным обеспечением, системными ресурсами (такие как процессорное время, память, сеть...) и экранирует разработчика от сложности программирования устройства, предоставляя программисту интерфейс для управления аппаратными средствами.
	\item Системные библиотеки содержат программные методы для разработчиков, необходимые для написания программного обеспечения для операционной системы. Они включают в себя методы создания и манипулирования, обработки файлов, сетевого программирования, и т.д. Это - жизненно важная часть операционной системы, так как вы не можете (или не  должны) связываться с ядром напрямую: библиотека экранирует системного программиста от сложности программирования ядра.
	\item Системные инструменты собраны с использованием системных библиотек и позволяют администраторам следить за системой: управлять процессами, перемещаться по файловой системе, запускать другие приложения, настраивать сеть\ldots
	\item Инструменты разработки предоставляют средства для сборки нового программного обеспечения в (или для) системе. Несмотря на то, что это необязательная часть операционной системы, мне очень нравится упоминать их, так как с Gentoo их наличие является требованием (почему дело обстоит так, мы узнаем позже). Эти инструменты включают в себя компиляторы (переводят код в машинный), компоновщики (собирают машинный код и объединяют его в рабочий двоичный файл) и средства, значительно упрощающие процесс сборки.
\end{itemize}

Другие находящиеся в системе библиотеки улучшают опыт написания кода разработчиками, обеспечивая доступ к методам, которые уже написаны другими. Примеры таких библиотек включают в себя графические (для управления окнами) или научные библиотеки. Их наличие не обязательно в каждой системе, но если вы захотите запустить определённый инструмент, это потребует установку соответствующих библиотек. Поверх этих дополнительных библиотек вы обнаружите готовые к установке инструменты конечного пользователя (пакеты офисных программ, мультимедийные утилиты, графические среды\ldots).


\end{document}